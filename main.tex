\documentclass[11pt]{article}
\usepackage{biblatex}
\addbibresource{refs.bib}

% --------- core formatting ---------
\usepackage[margin=1in]{geometry}
\usepackage{lmodern}
\usepackage[T1]{fontenc}
\usepackage{microtype}

% --------- math + figures ----------
\usepackage{amsmath,amsthm,amssymb,mathtools}
\usepackage{bm}
\usepackage{enumitem}
\usepackage{tikz}
\usetikzlibrary{arrows.meta,positioning}

% --------- citations & links -------
\let\bibhang\relax
\usepackage[numbers,sort&compress]{natbib}
\usepackage{hyperref}
\hypersetup{colorlinks=true,linkcolor=black,citecolor=black,urlcolor=black}

% --------- theorem machinery -------
\theoremstyle{plain}
\newtheorem{theorem}{Theorem}
\newtheorem{lemma}{Lemma}
\newtheorem{proposition}{Proposition}
\newtheorem{corollary}{Corollary}
\theoremstyle{definition}
\newtheorem{definition}{Definition}
\newtheorem{remark}{Remark}

% --------- notation ----------------
\newcommand{\N}{\mathbb{N}}
\newcommand{\R}{\mathbb{R}}
\newcommand{\Prb}{\mathbb{P}}
\newcommand{\E}{\mathbb{E}}
\newcommand{\io}{\text{\rm i.o.}}

% --------- title & author ----------
\title{\bf Ergodic Metric Drift as a Path–Validity Engine: Finite Graphs}
\author{Jacob Alexander Elliott}
\date{}

\begin{document}
\maketitle

\begin{abstract}
We study budgeted validity of finite paths in finite directed graphs under metric drift: at each time $t$ the edge–weight vector $w^{(t)}$ is redrawn, inducing a random path pseudometric $L_t$. A path $p$ is valid at $t$ if $L_t(p)\le B$ and its cylinders have a uniform measure floor. Under a \emph{uniform small–ball} condition for $L_t(p)$ and independence across $t$, the validity events occur infinitely often almost surely for every $p$; by countability, this holds simultaneously for all finite paths. The proof is a one–page Borel–Cantelli argument, but we push the envelope: we give sharp frequency and return–time laws; weak–dependence routes via Kochen–Stone; an online coupling that covers adaptive drifts; and a regeneration theorem for Harris–ergodic Markov drifts. We also treat quantile–adaptive budgets, non–additive path functionals, and finite–family coverage with coupon–collector bounds. A three–node $\mathrm{Exp}(\theta)$ example and a verification–centric framing (budgeted reachability under stochastic costs) complete a compact but rigorous note.
\end{abstract}

\paragraph{Keywords.}
Ergodic completeness; Borel–Cantelli; small–ball probabilities; regeneration; weak dependence; budgeted reachability; time–varying graphs.

% ------------------------------------------------------------------
\section{Introduction}
Let $G=(V,E)$ be a finite digraph. At each discrete time $t\in\N$ an edge–weight vector $w^{(t)}\in(0,\infty)^{|E|}$ is sampled and induces a path functional $L_t$ on finite paths $p=(e_1,\dots,e_k)$ via
\[
L_t(p)\;:=\;\sum_{i=1}^k w^{(t)}_{e_i}.
\]
Fix a budget $B>0$. A path is \emph{valid at time $t$} if $L_t(p)\le B$ and its cylinder events have $\mu_t$–measure at least a fixed floor $\tau>0$ (the floor guarantees measurability and a minimal probabilistic weight; it is otherwise inert in the proof).

Our central phenomenon is \emph{ergodic completeness (EC)}: under mild hypotheses, every finite path is valid infinitely often almost surely as the metric drifts. Operationally, EC is a robust budgeted–reachability guarantee in stochastic–cost verification, and a coverage guarantee for template sampling under dynamic costs. The mathematical core is classical: a uniform small–ball lower bound for $L_t(p)$, independence across $t$, and the second Borel–Cantelli lemma \cite[§2.3]{Billingsley,Durrett}. The rest is structure: frequency laws, weak–dependence, regeneration, and algorithmic implications.

% ------------------------------------------------------------------
\section{Setup, Standing Assumptions, and a Minimal Hypothesis}
Let $\mathcal{P}_{\rm fin}$ be the countable set of finite (directed) paths in $G$ (edges counted with multiplicity). For each $t$ let $\mathcal{F}_t$ be the cylinder $\sigma$–algebra on $V^\N$ and $\mu_t$ a measure with a fixed \emph{cylinder floor} $\tau>0$, i.e.\ every cylinder generated by a finite path has $\mu_t\ge\tau$.

\begin{definition}[Validity]
Given $B>0$ and $\tau>0$, a finite path $p$ is valid at time $t$ if $L_t(p)\le B$ and its cylinders lie in $\mathcal{F}_t$ with $\mu_t\ge\tau$.
\end{definition}

\begin{definition}[Ergodic completeness (EC)]
The system is \emph{ergodically complete} if for every $p\in\mathcal{P}_{\rm fin}$ the event $\{t:\; p\text{ is valid at }t\}$ is infinite almost surely.
\end{definition}

We isolate the exact hypothesis used by the proofs.

\medskip
\noindent\textbf{Uniform Small–Ball (USB).}
\emph{For each fixed finite path $p$ there exists $q_p\in(0,1]$ such that
\[
\inf_{t\ge 1}\;\Prb\big[L_t(p)\le B\big]\;\ge\; q_p.
\]
}

\begin{remark}[Where USB comes from]
USB holds if, for each $t$, the joint law of the edge weights along $p$ admits positive mass arbitrarily close to $0$—for example if the edge–weight vector has a density on $(0,\infty)^{|E|}$ that is locally integrable near $0$ in those coordinates. Independence across edges at fixed $t$ is \emph{not} required; we only need path–level small–ball probability.
\end{remark}

We assume throughout:
\begin{itemize}[leftmargin=1.5em]
\item[(A1)] USB holds for every fixed $p$ (with constant $q_p$ independent of $t$).
\item[(A2)] $\{w^{(t)}\}_{t\ge 1}$ are independent across $t$ (i.i.d.\ is convenient but not required).
\end{itemize}

% ------------------------------------------------------------------
\section{Main Result and Immediate Consequences}
\begin{theorem}[EC under USB + independence]\label{thm:EC}
Assume \textup{(A1)}–\textup{(A2)}. Then for each finite path $p$,
\[
\Prb\big[L_t(p)\le B~\io\big]=1,
\]
and, consequently, the system is ergodically complete almost surely.
\end{theorem}

\begin{proof}
Fix $p$. By USB there is $q_p>0$ with $\Prb[L_t(p)\le B]\ge q_p$ for all $t$. By independence across $t$, $\sum_t \Prb[L_t(p)\le B]=\infty$; the second Borel–Cantelli lemma yields $\Prb[L_t(p)\le B~\io]=1$ \cite[§2.3]{Billingsley,Durrett}. Since the measure floor holds by design, $p$ is valid infinitely often. Countability of $\mathcal{P}_{\rm fin}$ yields the simultaneous statement by a measure–one intersection.
\end{proof}

\begin{corollary}[All finite paths simultaneously]\label{cor:allpaths}
Under \textup{(A1)}–\textup{(A2)}, with probability one we have $L_t(p)\le B$ infinitely often for \emph{every} $p\in\mathcal{P}_{\rm fin}$ (hence EC holds).
\end{corollary}

\begin{remark}[0–1 law]
For fixed $p$, the tail event $\{L_t(p)\le B~\io\}$ has probability $0$ or $1$ by Kolmogorov’s $0$–$1$ law (independence across $t$). USB ensures positivity, hence the probability is $1$; BC2 is the standard, direct proof.
\end{remark}

\begin{proposition}[Frequencies and return times]\label{prop:freq}
Under \textup{(A1)}–\textup{(A2)} and i.i.d.\ across $t$, write $X_t:=\mathbf{1}\{L_t(p)\le B\}$ with $q_p:=\Prb[X_t=1]$.
Then: (i) $\tfrac1T\sum_{t\le T} X_t \to q_p$ a.s., with Hoeffding bounds
\(\Prb[\sum_{t\le T}X_t \le (q_p-\varepsilon)T] \le e^{-2\varepsilon^2 T}\) for $\varepsilon\in(0,1)$ \cite{Hoeffding};
(ii) the first validity time $\tau_p:=\min\{t\ge 1:\, X_t=1\}$ is geometric with
\(\Prb[\tau_p>m]=(1-q_p)^m\) and \(\E[\tau_p]=q_p^{-1}\).
\end{proposition}

\begin{proposition}[Finite families: a coupon–collector bound]\label{prop:family}
Let $S\subset \mathcal{P}_{\rm fin}$ be finite. Define $q_*:=\min_{p\in S} q_p$ and
\(\tau^\star:=\max_{p\in S}\tau_p\).
Then \(\Prb[\tau^\star > T] \le |S|(1-q_*)^T\) and \(\E[\tau^\star]\le \sum_{p\in S} q_p^{-1}\le |S|\,q_*^{-1}\).
\end{proposition}

\begin{remark}[Sharpness]
If $\underline{w}_e:=\inf \operatorname{supp}(w_e^{(t)})$ and $\sum_{e\in p} \underline{w}_e>B$, then
\(\Prb[L_t(p)\le B]=0\) for all $t$ and EC fails for $p$. Conversely, any model with USB satisfies Theorem~\ref{thm:EC}.
\end{remark}

% ------------------------------------------------------------------
\section{Lateral Routes Beyond Independence}
We record two robust mechanisms that keep EC intact when \textup{(A2)} is weakened.

\paragraph{Adaptive drifts with a uniform conditional success floor.}
Even if $w^{(t)}$ depends on the past, EC holds when the conditional success probability is uniformly bounded below.

\begin{proposition}[Online coupling]\label{prop:coupling}
Fix $p$ and suppose $\Prb(L_t(p)\le B\,|\,\mathcal{F}_{t-1})\ge q_p>0$ a.s.\ for all $t$, where
$\mathcal{F}_{t-1}=\sigma(w^{(1)},\dots,w^{(t-1)})$. Then $\Prb[L_t(p)\le B~\io]=1$.
\end{proposition}

\begin{proof}
Couple using i.i.d.\ $U_t\sim\mathrm{Unif}(0,1)$ and set $X_t:=\mathbf{1}\{U_t\le p_t\}$ with $p_t=\Prb(L_t(p)\le B\,|\,\mathcal{F}_{t-1})\ge q_p$. Define $Y_t:=\mathbf{1}\{U_t\le q_p\}$; then $Y_t\le X_t$ a.s.\ and $\{Y_t\}$ are i.i.d.\ Bernoulli$(q_p)$, hence $Y_t=1$ i.o.\ a.s., and so is $X_t$.
\end{proof}

\paragraph{Weak dependence via Kochen–Stone.}
Let $A_t:=\{L_t(p)\le B\}$. If $\sum_t \Prb(A_t)=\infty$ and correlations are controlled,
\[
\Prb(A_t~\io) \ \ge\ \limsup_{n\to\infty}\frac{\big(\sum_{t\le n}\Prb(A_t)\big)^2}{\sum_{i,j\le n}\Prb(A_i\cap A_j)}\!,
\]
hence $\Prb(A_t~\io)=1$ when the denominator grows as in the independent case \cite{KochenStone}. This covers a broad range of mixing drifts.

% ------------------------------------------------------------------
\section{Markov Drift: Regeneration Yields EC and Quantitative Bounds}
Let $\{w^{(t)}\}$ be a Harris–ergodic Markov chain on \((0,\infty)^{|E|}\). Assume a Doeblin minorization on blocks of length $t^\ast$: there exists $\beta\in(0,1]$ and a probability $\nu$ such that
\[
\Prb_x[w^{(t^\ast)}\in \cdot]\ \ge\ \beta\,\nu(\cdot)\qquad \text{for all states }x.
\]
Let $\{T_j\}$ be regeneration times of the split chain \cite[Ch.~10]{MeynTweedie}. Suppose there exists $q_0>0$ with
\[
\Prb\big[L_{T_j}(p)\le B \,\big|\, \text{regeneration at }T_j\big]\ \ge\ q_0 \quad \text{for all }j.
\]

\begin{theorem}[EC at regenerations $\Rightarrow$ EC on the full time axis]\label{thm:regen}
Under the conditions above, $\Prb[L_t(p)\le B~\io]=1$ for every finite path $p$. Moreover, with $\hat q:=\beta q_0$, the number of regeneration indices $j\le n$ with $L_{T_j}(p)\le B$ satisfies an SLLN with mean $\hat q$, and the regeneration–epoch waiting time has geometric tails with parameter $\hat q$.
\end{theorem}

\begin{proof}
At regenerations the chain i.i.d.\ re–enters with law $\nu$ with probability $\beta$; USB at regenerations gives success probability at least $q_0$ and i.i.d.\ Bernoulli trials with parameter $\hat q$. Borel–Cantelli on $\{T_j\}$ yields infinitely many successes a.s.; monotonicity transfers $\io$–occurrence to the full time axis.
\end{proof}

% ------------------------------------------------------------------
\section{Variants: Quantile Budgets and Non–Additive Path Functionals}
\paragraph{Quantile–adaptive budgets.}
Let $B_t:=c\cdot Q_\alpha(\mathsf{SP}_t)$ with $c>1$, where $\mathsf{SP}_t$ is the multiset of shortest–path lengths at time $t$ and $Q_\alpha$ an $\alpha$–quantile. If $\inf_t \Prb[L_t(p)\le B_t]\ge q_{p,\alpha,c}>0$, then all statements above hold with $B$ replaced by $B_t$.

\paragraph{Non–additive costs.}
Replace $L_t$ by a path functional $F_t$ (e.g.\ a max–edge, an $\ell^r$–norm, or a Lipschitz monotone transform) provided USB holds for $F_t(p)$. None of the arguments relies on additivity beyond USB and independence/regeneration.

% ===================== NEW MATERIAL BEGINS =====================

% ------------------------------------------------------------------
\section{Sharper Laws: LIL and Martingale Concentration}
\paragraph{Law of the iterated logarithm (LIL).}
Under i.i.d.\ across $t$, $X_t:=\mathbf{1}\{L_t(p)\le B\}\sim{\rm Bernoulli}(q_p)$ and the LIL yields
\[
\limsup_{T\to\infty}\frac{\sum_{t\le T}(X_t-q_p)}{\sqrt{2q_p(1-q_p)T\log\log T}}=1,\quad
\liminf_{T\to\infty}\frac{\sum_{t\le T}(X_t-q_p)}{\sqrt{2q_p(1-q_p)T\log\log T}}=-1
\]
almost surely; see, e.g., \cite[Thm.~1.5.1]{Durrett}. This quantifies the ultimate fluctuation scale around the mean frequency.

\paragraph{Freedman’s martingale inequality (adaptive drifts).}
Let $p_t:=\Prb(L_t(p)\le B\mid\mathcal{F}_{t-1})$ and $M_T:=\sum_{t\le T}(X_t-p_t)$ with conditional variance process $V_T:=\sum_{t\le T}\Var(X_t\mid\mathcal{F}_{t-1})\le \sum_{t\le T}p_t(1-p_t)$. Freedman’s inequality gives, for all $a,v>0$,
\[
\Prb\!\Big(M_T\ge a,\; V_T\le v\Big)\ \le\ \exp\!\Big(\!-\,\tfrac{a^2}{2(v+a/3)}\Big),
\]
hence uniform deviation bounds for adaptive success probabilities \cite{Freedman1975}. With a uniform floor $p_t\ge q_p$, one gets high‑probability lower bounds on cumulative successes and, therefore, on coverage latency.

% ------------------------------------------------------------------
\section{Weak Dependence via Mixing}
Write $\alpha(k)$ for the strong mixing coefficient of the process $\{X_t\}$ (or of $\{w^{(t)}\}$ through the indicator). If $\sum_{k\ge 1}\alpha(k)<\infty$ and $\inf_t\Prb(X_t=1)\ge q_p>0$, then
\[
\sum_{t=1}^n\sum_{s=1}^n \Prb(X_t=1,\ X_s=1)\ =\ \Theta\!\big((\sum_{t\le n}\Prb(X_t=1))^2\big),
\]
so Kochen–Stone applies and $\Prb[X_t=1\ \io]=1$ \cite{KochenStone}. This summability (and many variants) is classical; see \cite{Doukhan1994,Bradley2005} for surveys. More refined quantitative controls (e.g.\ Rio‑type Bernstein inequalities) can replace Hoeffding bounds when dependence is present.

\begin{remark}[Rosenblatt mixing and CLT]
Strong mixing with suitable rates also restores classical CLTs/LILs for $\sum(X_t-q_p)$, giving the same fluctuation hierarchies as in the i.i.d.\ case \cite{Rosenblatt1956,Bradley2005}.
\end{remark}

% ------------------------------------------------------------------
\section{Regeneration, Dobrushin Coefficients, and Minorization}
For a time‑homogeneous Markov drift $w^{(t+1)}\sim K(w^{(t)},\cdot)$ on a standard Borel space, geometric ergodicity follows from either a small‑set minorization (as used above) or from a contraction in the Dobrushin coefficient
\[
\delta(K)\ :=\ \sup_{x,y}\ \|K(x,\cdot)-K(y,\cdot)\|_{\rm TV}.
\]
If $\delta(K^{t^\ast})\le 1-\varepsilon$ for some $t^\ast$ (e.g., via a Doeblin condition), then total‑variation distances contract geometrically and one obtains regeneration with parameter $\beta\asymp\varepsilon$; see \cite[§16]{MeynTweedie} and the original coefficient \cite{Dobrushin1956}. This yields the same Bernoulli trials structure on regeneration epochs as in Theorem~\ref{thm:regen}.

% ------------------------------------------------------------------
\section{Model Classes Exhibiting USB (and When It Fails)}
\paragraph{Continuous weights with positive density near $0$.}
If each edge weight $w_e^{(t)}$ has a marginal density bounded below on $(0,\varepsilon)$ (uniformly in $t$), then for a $k$‑edge path $p$ there exists $c>0$ with $q_p\ge c\,\varepsilon^k$ by a direct product bound.

\paragraph{Log‑concave families.}
If $w^{(t)}$ has a log‑concave density on $(0,\infty)^{|E|}$ whose restriction to the coordinates of $p$ remains log‑concave and positive near the origin, then USB holds by the Prékopa–Leindler stability of small balls.

\paragraph{When USB fails.}
If $\inf\operatorname{supp}(w_e^{(t)})>\eta_e>0$ for some edges on $p$ and $\sum_{e\in p}\eta_e>B$, then $q_p=0$ for that $p$. In heterogeneous graphs this delineates which paths participate in EC.

% ------------------------------------------------------------------
\section{Design and Estimation for Finite Path Families}
Let $S\subset\mathcal{P}_{\rm fin}$ be finite. With i.i.d.\ across $t$, the uniform Hoeffding bound gives the sample size for estimating all $q_p$ within $\varepsilon$ with failure probability $\delta$:
\[
T\ \ge\ \frac{1}{2\varepsilon^2}\,\log\frac{2|S|}{\delta}.
\]
This controls the empirical prioritization of witnesses by estimated latency $1/\hat q_p$. Under regeneration, the same bound applies to the Bernoulli sequence observed on $\{T_j\}$, with $T$ replaced by the number of regeneration epochs.

% ------------------------------------------------------------------
\section{Additional Examples}
\paragraph{Subexponential drifts.}
If $w_e^{(t)}$ are i.i.d.\ with densities behaving like $f_e(x)\asymp x^{\alpha_e-1}$ near zero ($\alpha_e>0$), then for a $k$‑edge path $p$ the small‑ball $\Prb[L_t(p)\le B]$ scales like $B^{\sum_{e\in p}\alpha_e}$ as $B\downarrow 0$ (by convolution at the origin), giving explicit $q_p$ lower bounds for small budgets.

\paragraph{Mixtures with a point mass at $0$.}
If each $w_e^{(t)}$ has a positive atom at $0$ (e.g.\ with probability $\rho_e>0$), then $q_p\ge \prod_{e\in p}\rho_e$ for all budgets $B>0$, and EC holds with fast latencies.

% ------------------------------------------------------------------
\section{Complexity, Estimation, and Verification Framing}
\paragraph{Validity checking.}
Given $w^{(t)}$ and $p=(e_1,\dots,e_k)$, computing $L_t(p)$ is $O(k)$; the cylinder–floor is a fixed design constraint on $\mu_t$. EC is a process–level almost–sure statement, not a finite–time worst–case guarantee.

\paragraph{Estimating success rates.}
With i.i.d.\ across $t$, $\hat q_{p,T}:=T^{-1}\sum_{t\le T}\mathbf{1}\{L_t(p)\le B\}$ is the MLE for $q_p$; Hoeffding–type concentration gives nonasymptotic CIs \cite{Hoeffding}. Under regeneration, estimate $\hat q$ at $\{T_j\}$.

\paragraph{Budgeted reachability (model checking).}
EC supplies a universal recurrence under drift: every finite execution template hits budget infinitely often. For finite families of witnesses, Proposition~\ref{prop:family} gives explicit coverage and latency bounds.

% ------------------------------------------------------------------
\section{Related Work}
Our arguments use standard Borel–Cantelli tools \cite[§2.3]{Billingsley,Durrett}, the Kochen–Stone bound for weak dependence \cite{KochenStone}, and regeneration for Harris chains \cite{MeynTweedie}. In the dynamic–networks literature \cite{TVGSurvey}, temporal connectivity and schedule constraints dominate; our contribution inserts a random metric and a budget–validity layer, with an almost–sure recurrence guarantee under minimal small–ball hypotheses. Mixing frameworks and deviation inequalities are surveyed in \cite{Doukhan1994,Bradley2005}; Rosenblatt’s original strong‑mixing paper gives a classical route to CLTs and LILs \cite{Rosenblatt1956}. On regeneration and quantitative ergodicity for Markov chains, see also \cite{Nummelin1984,Asmussen2003,Dobrushin1956}.

% ------------------------------------------------------------------
\section*{Acknowledgments}
The finite–case backbone was intentionally kept independent of modal/projection layers and other idioms, in service of a clean, verifiable proof.

\bibliographystyle{plainnat}
\bibliography{refs}

\end{document}
