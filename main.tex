% !TeX program = pdflatex
% !BIB program = biber
\documentclass[11pt]{article}

% Encoding & fonts
\usepackage[utf8]{inputenc}
\usepackage[T1]{fontenc}
\usepackage{lmodern}

% Layout & math
\usepackage[margin=1in]{geometry}
\usepackage{amsmath,amsthm,amssymb,mathtools}
\usepackage{dsfont,bm}
\usepackage{tikz-cd}
\usepackage{enumitem}
\usepackage{microtype}

% Links & cross-references
\usepackage[hidelinks]{hyperref}
\usepackage[nameinlink,capitalise,noabbrev]{cleveref}

% Bibliography (biblatex + biber)
\usepackage[
  backend=biber,
  style=alphabetic,
  doi=false,isbn=false,url=false,eprint=false,
  maxnames=4
]{biblatex}
\addbibresource{refs.bib} % keep refs.bib alongside this main.tex

\title{\textbf{Ergodic Metric Drift as a Path Validity Engine}\\[0.25em]\large Finite Graphs and Non-Reversible Advantage}
\author{}
\date{\today}

% --- Theorem setup ---
\theoremstyle{plain}
\newtheorem{theorem}{Theorem}[section]
\newtheorem{lemma}[theorem]{Lemma}
\newtheorem{proposition}[theorem]{Proposition}
\newtheorem{corollary}[theorem]{Corollary}
\newtheorem{conjecture}[theorem]{Conjecture}

\theoremstyle{definition}
\newtheorem{definition}[theorem]{Definition}
\newtheorem{assumption}[theorem]{Assumption}
\newtheorem{example}[theorem]{Example}

\theoremstyle{remark}
\newtheorem{remark}[theorem]{Remark}
\newtheorem{notation}[theorem]{Notation}

% --- Macros ---
\DeclareMathOperator*{\argmax}{arg\,max}
\DeclareMathOperator*{\argmin}{arg\,min}
\DeclareMathOperator{\KL}{\mathsf{KL}}
\DeclareMathOperator{\TV}{\mathsf{TV}}
\DeclareMathOperator{\Var}{\mathsf{Var}}
\DeclareMathOperator{\Cov}{\mathsf{Cov}}
\DeclareMathOperator{\diag}{\mathsf{diag}}
\DeclareMathOperator{\supp}{\mathsf{supp}}
\newcommand{\EE}{\mathbb{E}}
\newcommand{\PP}{\mathbb{P}}
\newcommand{\RR}{\mathbb{R}}
\newcommand{\NN}{\mathbb{N}}
\newcommand{\1}{\mathds{1}}

\newcommand{\U}{\mathcal{U}}
\newcommand{\X}{\mathcal{X}}
\newcommand{\A}{\mathcal{A}}

\newcommand{\Free}{\mathsf{F}}
\newcommand{\EP}{\sigma}
\newcommand{\Budget}{\mathsf{B}}
\newcommand{\Cost}{\mathsf{Cost}}
\newcommand{\Gain}{\mathsf{IG}}

% Cross-series shorthands
\newcommand{\Proj}{\mathsf{Proj}}
\newcommand{\Nec}{\Box}
\newcommand{\Poss}{\Diamond}
\newcommand{\FlowCent}{\mathsf{FC}}
\newcommand{\QFI}{\mathsf{QFI}}

\begin{document}
\maketitle

\begin{abstract}
We package \emph{ergodic metric drift} into a path-validity engine for inference on finite graphs. A Foster--Lyapunov drift together with a small-set/minorization guarantee yields geometric ergodicity, explicit mixing bounds, and exponential tails for hitting times. In this runtime, drift/minorization are expressed as compile-time contracts and schedulable budgets; non-reversibility (positive entropy production) improves contraction and suppresses cyclical stalling. Our analysis draws on classical regeneration and small-set techniques \cite{MeynTweedie2009,Nummelin1984,RobertsRosenthal2004}, Dobrushin contraction \cite{Seneta2006,LevinPeresWilmer2009}, path coupling \cite{BubleyDyer1997}, conductance/spectral proxies \cite{LevinPeresWilmer2009}, and curvature-based contraction \cite{Ollivier2009,JoulinOllivier2010}. We also give computable certificates and reference implementations (Mathematica) and formal scaffolding (Lean/Coq). The engine integrates with \emph{Parallel Murphy Budgeting}, \emph{Idiom Projections}, \emph{Measure-Theoretic Flow Centrality}, and \emph{Large~A}, and satisfies the \emph{Non-Reversibility Selector} constraints posited elsewhere in the series \cite{MurphyBudget2025,IdiomProjections2025,FlowCentrality2025,GradedEffects2025,NRSinPrep2025,QFI2025}.
\end{abstract}

\section{Introduction}
Inference over finite graphs---random walks, lazy diffusions, controlled kernels---underlies the runtime's transport layer. \emph{Ergodic metric drift} is the minimal certification that mass transport is covering, stable, and improvable: a Lyapunov descent outside a small set and a minorization on that set. The payoffs are standard yet strong: geometric total-variation (TV) convergence, regeneration, and tail bounds for hitting times \cite{MeynTweedie2009,RobertsRosenthal2004,Nummelin1984}.

\paragraph{Non-reversible advantage.}
When $P$ is non-reversible (positive entropy production),
\[
\EP(P,\pi)\;=\;\sum_{x,y}\pi(x)P(x,y)\log\frac{\pi(x)P(x,y)}{\pi(y)P(y,x)}\;\ge 0,
\]
strict positivity ($\EP>0$) aligns inference with directional information flow. Empirically and theoretically this often sharpens contraction compared to the reversible proxy (see the pseudo-spectral and curvature discussions in \cite{LevinPeresWilmer2009,Ollivier2009,JoulinOllivier2010}). In our runtime, $\EP>0$ is the operative form of the \emph{Non-Reversibility Selector (NRS)} hypothesis \cite{NRSinPrep2025}.

\subsection{Contributions}
\begin{enumerate}[label=(C\arabic*)]
\item A finite-graph formulation of ergodic metric drift with small-set/minorization leading to geometric ergodicity (\cref{thm:drift_minorization_geometric}); cf.\ \cite{MeynTweedie2009,RobertsRosenthal2004,Nummelin1984}.
\item Three computable certificates: Dobrushin/TVD contraction (\cref{prop:dobrushin}), conductance/spectral proxies (\cref{prop:conductance}, \cref{prop:spectral}), and curvature/path-coupling templates (\cref{thm:path_coupling}); cf.\ \cite{Seneta2006,LevinPeresWilmer2009,BubleyDyer1997,Ollivier2009,JoulinOllivier2010}.
\item A regeneration-based hitting-time guarantee that yields \emph{path validity} (\cref{prop:pathvalidity}); cf.\ \cite[Ch.~15]{MeynTweedie2009} and \cite{RobertsRosenthal2004}.
\item Implementation-grade routines (Mathematica) that verify drift/minorization, compute Dobrushin and entropy production, attempt minorization discovery, and implement monotone CFTP \cite{ProppWilson1996}; and formal scaffolds in Lean/Coq capturing kernels, drift data, and small sets.
\end{enumerate}

\section{Preliminaries}
Let $G=(V,E)$ be a finite directed graph (loops allowed). A \emph{Markov kernel} $P$ is a row-stochastic matrix supported by $E\cup\{(x,x)\}$. For distributions $\mu,\nu$, $\TV(\mu,\nu)=\tfrac{1}{2}\sum_{x\in V}|\mu(x)-\nu(x)|$ \cite{LevinPeresWilmer2009}. A stationary distribution $\pi$ satisfies $\pi P=\pi$.

\begin{definition}[Foster--Lyapunov drift and small set]\label{def:driftSmall}
Let $V:V\to[1,\infty)$ and constants $\lambda\in[0,1)$, $b<\infty$. $P$ satisfies \emph{geometric drift} if
\[
\EE[V(X_{t+1})\mid X_t=x]\le \lambda V(x)+b\quad\forall x\in V.
\]
A set $C\subseteq V$ is \emph{small} if there are $m\in\NN$, $\epsilon>0$, and a probability measure $\nu$ with
$P^m(x,\cdot)\ge \epsilon\,\nu(\cdot)$ for all $x\in C$ (\emph{minorization}). See \cite{Nummelin1984,MeynTweedie2009}.
\end{definition}

\begin{remark}
On finite graphs, irreducibility and aperiodicity already imply geometric convergence, but the drift/small-set pair exposes quantitative constants and interfaces with budgeting/contracts in our runtime \cite{MeynTweedie2009,RobertsRosenthal2004}.
\end{remark}

\section{Drift + minorization \texorpdfstring{$\Rightarrow$}{⇒} geometric ergodicity}
\begin{theorem}[Geometric ergodicity via drift and small set]\label{thm:drift_minorization_geometric}
If $P$ is $\pi$-irreducible and aperiodic, and satisfies \cref{def:driftSmall}, then there exist $R<\infty$ and $\rho\in(0,1)$ such that
\[
\TV\!\big(\delta_x P^t,\pi\big)\le R\,V(x)\,\rho^t\quad\forall x\in V,\ \forall t\in\NN.
\]
\textit{Claim-specific citations:} This is the standard Foster--Lyapunov $+$ small-set conclusion via Nummelin splitting and regeneration; see \cite[Thm.~15.0.1]{MeynTweedie2009}, \cite{Nummelin1984}, and \cite[Sec.~2]{RobertsRosenthal2004}.
\end{theorem}

\subsection{Computable TV contraction: Dobrushin and curvature}
\begin{definition}[Dobrushin’s ergodicity coefficient]
$\alpha(P)=1-\sup_{x,x'}\tfrac{1}{2}\sum_y |P(x,y)-P(x',y)|$. Then
\[
\TV(\mu P,\nu P)\le (1-\alpha(P))\,\TV(\mu,\nu)\quad\forall\mu,\nu.
\]
\end{definition}

\begin{proposition}[One-step TV contraction]\label{prop:dobrushin}
If $\alpha(P)>0$, then $\TV(\mu P^t,\nu P^t)\le (1-\alpha(P))^t \TV(\mu,\nu)$. \textit{Cites:} \cite[Ch.~4]{Seneta2006,LevinPeresWilmer2009}.
\end{proposition}

Beyond TV, coarse Ricci curvature $\kappa$ of $P$ on a metric space contracts $W_1$ distances: if $\kappa>0$ then $W_1(\mu P,\nu P)\le (1-\kappa)W_1(\mu,\nu)$ \cite{Ollivier2009,JoulinOllivier2010}. On graphs, path metrics and path coupling (\cref{thm:path_coupling}) are discrete analogues.

\subsection{Conductance and spectral proxies}
\begin{proposition}[Conductance proxy]\label{prop:conductance}
For a reversible chain with conductance $\Phi$, the mixing time satisfies $t_{\mathrm{mix}}(\varepsilon)\le \Phi^{-2}\,\log\!\big(\varepsilon^{-1}\pi_{\min}^{-1}\big)$ up to universal constants; equivalently, $\Phi^2\lesssim 1-\lambda_\star\lesssim \Phi$ \cite[Chs.~7--13]{LevinPeresWilmer2009}.
\end{proposition}

\begin{proposition}[Spectral gap proxy]\label{prop:spectral}
If $P$ is reversible w.r.t.\ $\pi$ with second eigenvalue modulus $\lambda_\star<1$,
\[
\TV(\delta_x P^t,\pi)\le \tfrac12 \sqrt{\frac{1-\pi_{\min}}{\pi_{\min}}}\,\lambda_\star^t.
\]
\textit{Cites:} \cite[Ch.~12]{LevinPeresWilmer2009}. For non-reversible chains replace $\lambda_\star$ by a pseudo-spectral proxy; see discussion in \cite[§19]{LevinPeresWilmer2009}.
\end{proposition}

\subsection{Path coupling and exact sampling}
\begin{theorem}[Path coupling]\label{thm:path_coupling}
Let $d$ be a path metric on $V$ and suppose a one-step coupling $\mathcal{C}$ satisfies
$\EE[d(X_1,Y_1)\mid X_0=x,Y_0=y]\le \beta\,d(x,y)$ for all adjacent pairs with $\beta<1$.
Then $\TV(\delta_x P^t,\delta_y P^t)\le \EE[d(X_t,Y_t)]\le \beta^t d(x,y)$, hence geometric mixing.
\textit{Cites:} \cite{BubleyDyer1997,LevinPeresWilmer2009}.
\end{theorem}

\begin{remark}[Exact sampling]
Monotone/dominated couplings enable \emph{coupling from the past} (CFTP) for perfect sampling from $\pi$ \cite{ProppWilson1996}. Our implementation provides a monotone CFTP template.
\end{remark}

\section{Path validity and hitting times}
\begin{definition}[Path validity]
$P$ has \emph{path validity} if for every $G\subseteq V$, the hitting time $\tau_G=\inf\{t\ge 0:X_t\in G\}$
has uniformly bounded mean and exponential tail: $\sup_x\EE_x[\tau_G]<\infty$ and
$\sup_x\PP_x(\tau_G>t)\le C\rho^t$ for some $C<\infty$, $\rho\in(0,1)$.
\end{definition}

\begin{proposition}[Regeneration $\Rightarrow$ path validity]\label{prop:pathvalidity}
Under \cref{thm:drift_minorization_geometric}, the regeneration structure implies exponential tails for return/hitting times and hence path validity, with constants depending on $(\lambda,b,m,\epsilon)$ and $G$ via the minorization on $C\cup G$. \textit{Cites:} \cite[Ch.~15]{MeynTweedie2009}, \cite{RobertsRosenthal2004}.
\end{proposition}

\section{Runtime alignment and the NRS selector}
\paragraph{NRS (\texorpdfstring{$\EP>0$}{EP>0}).} Positive entropy production correlates with stronger contraction (via curvature or pseudo-spectral proxies) and shorter certified horizons; we treat $\EP$ as a first-class index in compile-time contracts \cite{NRSinPrep2025}.

\paragraph{Murphy budgeting.} Truncations at certified $t_{\rm mix}$ give safe horizons; greedy scheduling over submodular gains achieves $(1-1/e)$-approximation \cite[Ch.~6]{LevinPeresWilmer2009}, aligning with \cite{MurphyBudget2025}.

\paragraph{Contracts.} Graded effects annotate $(\lambda,b,m,\epsilon,\alpha,\Phi,\EP)$ as compile-time indices; kernels failing thresholds are rejected at type-check time \cite{GradedEffects2025}.

\section{Worked examples}
\begin{example}[Lazy cycle]
On $C_n$ with $P=\tfrac12 I+\tfrac14(\text{left}+\text{right})$, reversibility holds with spectral gap $1-\cos\!\tfrac{2\pi}{n}$; \cref{prop:spectral} and \cref{thm:path_coupling} give matching rates up to constants \cite{LevinPeresWilmer2009,BubleyDyer1997}.
\end{example}

\begin{example}[Directed star with stickiness]
Let $V=\{0,1,\ldots,k\}$ with $i\to 0$ and $0\to i$ edges; add a self-loop of weight $\gamma\in(0,1)$ at $0$. Minorization at $\{0\}$ is immediate; a piecewise-constant $V$ verifies drift and \cref{thm:drift_minorization_geometric}.
\end{example}

\section*{Acknowledgments and cross-references}
This paper interlocks with \emph{QFI}, \emph{Idiom Projections}, \emph{Modal Shell}, \emph{Parallel Murphy Budgeting}, \emph{Measure‑Theoretic Flow Centrality}, \emph{Large A}, and \emph{NRS} \cite{QFI2025,IdiomProjections2025,ModalShell2025,MurphyBudget2025,FlowCentrality2025,GradedEffects2025,NRSinPrep2025}.

% Print all entries while editing; remove once every claim is cited in-text.
\nocite{*}
\printbibliography

\end{document}